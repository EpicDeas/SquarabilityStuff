
\documentclass[11pt]{exam}
\usepackage[utf8]{inputenc}
\usepackage[IL2]{fontenc}
\usepackage[czech]{babel}
\usepackage{amsmath,amsfonts,amssymb,amsthm} 
\usepackage{hyperref}
\usepackage{indentfirst}

\setlength\parindent{0pt} 

\DeclareMathOperator{\s}{S}

\newcommand{\R}{\mathcal{R}}

\usepackage{xfrac}
\usepackage{graphicx}

\newcommand{\mysection}[1]{%
  \par
  \pagebreak[2]%
  \refstepcounter{section}%
    \everypar={%
      {\setbox0=\lastbox}% Remove the indentation
      \addcontentsline{toc}{section}{%
        {\protect\makebox[0.3in][r]{\thesection.} \hspace*{3pt}#1}}%
      \textbf{\thesection\space\space{#1}. }%
      \everypar={}%
    }%
  \ignorespaces
}



\def\title#1{\medskip\centerline{\Large \textbf{#1}}}
\def\endtitle{\centerline{\today}\par\medskip}


\begin{document}

\title{Variants of squarability problem}
\endtitle

In all versions let $\R$ denote a set of axis-parallel rectangles in $\mathbb{R}^2$ and $\s$ be a mapping from $\R$ to axis-parallel squares in $\mathbb{R}^2$ satisfying certain restrictions  (note that $\s$ might not exist). Then $\s(\R)$ is a set of squares obtained from $\R$ in a way specific to the particular variant and $\s(R)$ is the square representing the rectangle $R\in\R$. In each variant we explain the restrictions placed on the input set of rectangles $\R$ and on the output set of squares $\s(\R)$.\\

There are four intersection types: Corner intersection, side-piercing, cross intersection and containment. We say $\R$ and $\s(\R)$ are \textit{combinatorially equivalent} if the intersection types are preserved and these intersection happen exactly on the same sides (and corners). For example, if $R_1,R_2\in\R$ have corner intersection that is in upper left corner on $R_1$ and lower right corner of $R_2$, the same must hold for $\s(R_1)$ and $\s(R_2)$.\\

In all the variants here, we assume that the input set $\R$ contains no two rectangles with side-piercing or cross intersections.\\

\mysection{Preserve order of all sides}
The output $\s(\R)$ has to be combinatorially equivalent to $\R$ and the respective order of sides on both axes has to be preserved. On a chosen axis, we can contruct the sequence of sides of rectangles $\R$ from left to right as they appear, i.e. every rectangle will appear exactly twice. Then the same sequence of sides has to be realized in $\s(\R)$.

\mysection{Combinatorial equivalence}
The output $\s(\R)$ has to be combinatorially equivalent.\\

We say that $\R$ and $\s(\R)$ are \textit{order equivalent}, if the cyclical order of intersecting rectangles around each rectangle is preserved in their images. Moreover, they are \textit{fixed-north-order equivalent} if the sequences of sides of intersecting rectangles constructed for each rectangle by starting in the upper right corner and walking counter-clockwise around the border of the rectangle and recording every side are also preserved in their images.\\

\mysection{Fixed north} We require that $\R$ and $\s(\R)$ are fixed-north-order equivalent.

\mysection{Preserve order} We require that $\R$ and $\s(\R)$ are  order equivalent.

\mysection{Intersection types} The output $\s(\R)$ has to keep the intersection types.\\

We say that $\R$ and $\s(\R)$ are \textit{intersection-pattern equivalent} if the following holds: $R_1\cap R_2\neq\emptyset$ if and only if $\s(R_1)\cap\s(R_2)\neq\emptyset$ for all $R_1,R_2\in\R$. Note that both combinatorial equivalence and order equivalence imply intersection-pattern equivalence.\\

\mysection{Keep intersections, forbid side-piercing} We require intersection-pattern equivalence. Squares in the output set $\s(\R)$ must only have corner intersections or containment.

\mysection{Keep intersections, allow side-piercing} We only require intersection-pattern equivalence.



















\end{document} 
